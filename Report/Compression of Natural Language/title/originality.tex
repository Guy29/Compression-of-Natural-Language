\newpage
\thispagestyle{empty} % Optional: to ensure the page does not have headers or footers


\begin{center}
    \vspace*{\stretch{1}}
    \textbf{\Large Summary}
    \vspace{0.5cm}
\end{center}

\noindent Human language contains many redundancies and regularities. It is possible, for example, for a person to infer from an incomplete sentence a missing letter or word, or to spot an error through an inconsistency between text and context.

\noindent This is done, first, through the abstraction of an underlying, concise representation of the text, and then through a re-generation of the elaborated text from the conceptual understanding. In humans, one's ability to induce a general pattern and deduce its application to a specific case in this way is often an indication of their comprehension of a text, and of having a grasp of the underlying meaning.

\noindent Because of this, it is believed by some researchers (most notably Marcus Hutter) that the ideal lossless compression of a text would require comprehension, and that the first is therefore an AI-complete problem.

\noindent This project aims to explore the relationship between compression and comprehension.

\noindent To view the code used in this project, visit \url{https://github.com/Guy29/FYP}.


\begin{center}
    \vspace*{\stretch{1}}
    \textbf{\Large Statement of Originality}
    \vspace{1cm}
\end{center}

\noindent This report is submitted as part requirement for the degree of BSc Computer Science at the University of Sussex. It is the product of my own labour except where indicated in the text.

\begin{flushright}

\begin{tabular}{rl}
    Signed: & Guy Aziz \\
    Candidate number: & 267649 \\
    Date: & \today
\end{tabular}

\end{flushright}



\vspace*{\stretch{2}}

\clearpage
