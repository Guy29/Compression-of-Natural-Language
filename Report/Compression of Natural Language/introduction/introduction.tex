\chapter{Introduction}

Human language contains many redundancies and regularities. It is possible, for example, for a person to infer from an incomplete sentence a missing letter or word, or to spot an error through an inconsistency between text and context.

This is done, first, through the abstraction of an underlying, concise representation of the text, and then through a re-generation of the elaborated text from the conceptual understanding. In humans, one's ability to induce a general pattern and deduce its application to a specific case in this way is often an indication of their comprehension of a text, and of having a grasp of the underlying meaning.

Because of this, it is believed by some researchers (most notably Marcus Hutter) that the ideal lossless compression of a text would require comprehension, and that the first is therefore an AI-complete problem.

This project aims to explore the relationship between compression and comprehension.

To view the code used in this project, visit \url{https://github.com/Guy29/FYP}.

\section{Motivation}
\label{subsec:motivation}

Occam’s razor is a general principle in science and rationality commonly attributed to William of Ockham (1290?-1349?) which states “entities should not be multiplied beyond necessity”, usually interpreted to mean “the simplest theory is usually the correct one.”

It is easy to take this principle for granted, as it seems to work in both science and in our daily experience. But why this should be the case, i.e. why we live in a regular enough world that simpler models of it tend to be more correct than more complex ones, is not obvious, and it has been historically pointed out by many philosophers (most famously David Hume) that knowledge gained in this way stands on shaky foundations. \autocite{Henderson2018}

Solomonff’s theory of inductive inference provides a formalism for Occam’s razor. It posits an agent making observations in a world that operates by an unknown algorithm, and based on that premise shows that the agent would do well to assume that the length of the algorithm by which its world runs in effect follows a probability distribution that assigns shorter (and therefore simpler) algorithms more probability. This probability distribution is known as the universal prior.

The argument Solomonoff uses can be understood as follows: if the agent considers all algorithms of the same complexity as equally likely, then the algorithms can be divided into subsets where each subset is functionally equivalent. Because there are more ways to implement simpler algorithms than more complex ones, they accrue more probability mass. For example, a crime investigator who knows that Alice likes apple pie and Bob doesn’t may consider the following hypotheses (of equal complexity) for the disappearance of an apple pie:

\begin{enumerate}
    \item Alice stole it, wearing a red shirt.
    \item Alice stole it, wearing a blue shirt.
    \item Bob stole it, after having a change of taste.
\end{enumerate}

If each of these hypotheses is given the same probability initially, based on being of equal complexity, then a grouping of the first two as functionally equivalent (the colour of the shirt being irrelevant) makes it twice as likely that Alice is the culprit.

Solomonoff’s theory of inductive inference uses the concept of Kolomogorov complexity, which refers to the length of the shortest program that would produce a specific output. For example, the Kolomogorov complexity of the string “1111 … 11111” is lower than that of “wp9j8 … fd27c”, as the first is more regular.

In the same way that Occam’s razor lacked formalism and proof until Solomonoff, the concept of intelligence similarly lacks formalism in modern computer science, and psychologist R. J. Sternberg remarks “there seem to be almost as many definitions of intelligence as there were experts asked to define it.” \autocite{Legg2007}

Marcus Hutter \autocite{Hutter2000} proposes a formalism of an intelligent agent which he terms AIXI that combines the above ideas as well as ideas from reinforcement learning. AIXI is a theoretical agent which, in each time step,

\begin{enumerate}
    \item Makes an action $a_i$.
    \item Receives an observation $o_i$ and a reward (positive or negative) $r_i$.
    \item Generates all possible algorithms by which its world can run which would have predicted all of its observations and rewards so far, and weighs the probabilities of those algorithms inversely to their length (i.e. it applies the universal prior).
    \item Uses the most likely algorithms (or models of its world) to simulate the world, and thereby decide on its following actions to maximize its reward.
\end{enumerate}

It can be seen that the above description of AIXI requires an agent that can effectively create a concise, compressed model of the observations it has made of its world so far, and that having such a model is most predictive of its success. For this reason, Hutter believes that intelligence, adaptability, and data compression are tightly bound.

\section{Professional and Ethical Considerations}
\label{subsec:bcs}
To the best of my knowledge, there are no professional or ethical considerations, as given by the BCS code of conduct (\url{https://www.bcs.org/membership/become-a-member/bcs-code-of-conduct/}) that would constrain any part of this project. The core of the project is a review and discussion of existing techniques, and practical applications will likely be limited to improvements on compression algorithms. All data used in this project is in the public domain.



\section{Theory}
\label{sec:theory}

\subsection{Shannon}
\label{subsec:shannon}

In his seminal 1948 paper "A Mathematical Theory of Communication", Claude Shannon introduced the concept of information theoretic entropy \autocite{shannon1948mathematical}, defining it as the average amount of information provided by a stochastic source of data, given by the equation

\[H(X) = \mathbb{E}[-\log p(X)] = -\sum_{x \in \mathcal{X}} p(x) \log p(x)\]

for some discrete random variable X. In simple terms, the entropy of a variable is the average number of yes/no questions one needs to ask to determine the value of that variable. For a fair coin, one needs to ask only one yes/no question to learn the outcome, and so the entropy of that outcome is 1 bit (sometimes called 1 shannon).

For the English language, the entropy of individual letters based on their frequency is known to be approximately 4.14 bits. This can be obtained by calculating the negated sum of the products of the frequencies of the letters with the logs of those frequencies, in line with the formula above.

In his paper on the entropy of English text, Shannon improved on this, estimating that - when the letter is considered within its context - English contains 0.6 – 1.3 bits of information per letter \autocite{Shannon1951}. He arrives at this estimate by asking subjects to reconstruct a text by guessing its letters one by one, only being told whether their guess is correct or not, and by noting how well they perform at this task given a varying amount of context (1 to 100 previous letters). He finds that, when the subjects are given 100 previous letters for context, they are able to guess the next letter on their first try 80\% of the time. 

Shannon's analysis provides a theoretical best-case limit on how well natural language can be compressed. 

\subsection{Montague grammar}
In 1970, logician Richard Montague posited that there is “no important theoretical difference between natural languages and the artificial languages of logicians” and that it is “possible to comprehend the syntax and semantics of both kinds of language within a single natural and mathematically precise theory.” \autocite{Montague1970}

Montague gives a grammar which can be used to parse English, and a specification for a parser based on this grammar is given by Friedman \& Warren \autocite{Friedman1978}.
\section{Practice}
\label{subsec:practice}

\subsection{Tools}

\todo[inline, color=yellow]{For each off these, either relate to the report, move to the section in the report where it's relevant, or delete.}

A variety of compression algorithms are in use today, many of which have been applied to natural language.

\textcite{Mahoney2011} has compiled a document which lists many compression program benchmarks on the \texttt{enwik9} challenge, along with the algorithms and techniques used by each, which is especially relevant to this topic. I touch briefly on each of these below.

\subsubsection{Lempel Ziv and its variations}
These include the original LZ77 \autocite{Ziv1977} and LZ78 \autocite{Ziv1978} algorithms.

\textbf{LZ77}

The first (LZ77) works by replacing repeated substrings inside a larger string with a reference to their first occurrence, indicated by a distance D from the current position at which that occurrence starts and the length L of characters that should be copied from that offset.

For example, the underlined part of the string “CC\textbf{BAABABBBA\underline{B}}\underline{AABABBBAB}BABB” could be represented by (D=9, L=10), indicating that the beginning of the string “CCBAABABBBA” is followed by a 10-character long sequence which is copied from an offset that starts 9 characters back (indicated above in bold).

\textbf{LZ78}

The second algorithm (LZ78) works by incrementally building up a dictionary of previously seen substrings, each composed of a pair of (substring seen even earlier in the text, plus one additional character).

For example, the string “AABABBBABAABABBBABBABB” would be divided into the sections

\begin{center}
\textbf{A}|A\textbf{B}|AB\textbf{B}|\textbf{B}|AB\textbf{A}|ABA\textbf{B}|B\textbf{B}|ABB\textbf{A}|BB
\end{center}

each of which is a previously seen substring plus one additional character (marked in bold). If the sections are numbered 1, 2, 3, etc., the whole string can be abbreviated to

\begin{center}
\textbf{A}|1\textbf{B}|2\textbf{B}|0\textbf{B}|2\textbf{A}|5\textbf{B}|4\textbf{B}|3\textbf{A}|7
\end{center}

In this representation, 1B simply means “the contents of section 1, plus a B”.


\subsubsection{Symbol Ranking}
The SR family of algorithms each keep a ranking of potential symbols, usually as a probability density function. For example, it could estimate a 0.1 probability that a randomly chosen letter is “e” and a 0.01 probability that it is “j”, and consequently assign “e” a shorter code.

This can be done in a rolling fashion where the encoder will update the ranking of the symbols based on how frequent they have been in the text in a sliding window ending at the current symbol, and similarly the decoder will update its own ranking and codebook identically as it is decoding the text. In this case, the codebook keeps changing throughout the text, but both encoder and decoder can construct identical ones based on the text seen so far. 

\subsubsection{Prediction by Partial Matching}
PPM is a SR-based algorithm augmented by conditioning the probabilities of symbols on their “context”, consisting for example of the last n characters read.

If the algorithm has so far read the symbols “aard”, for example, it may estimate a 0.9 probability of the next symbol being “v” and a 0.01 probability of the next symbol being “x”. Based on these estimates, it assigns each of the characters “v” and “x” codes of lengths that correspond to how surprising they would be and will encode the actually observed character using that code.

In decompression, an identical predictor will read the encoded text, create the same codebook, and then use the recorded symbol code to reconstruct the text. \autocite{Fenwick1998}

\subsubsection{Burrows Wheeler Transform}
This algorithm, due to Michael Burrows and David Wheeler \autocite{Burrows1994} acts as an aid to other algorithms which perform better when fed data with runs of repeated characters. The algorithm outputs a reversible permutation of the characters of a string which contains more such runs.

The forward transform works as follows: first, an EOF character is appended to the string, then all circular shifts of the string are calculated and added to an array. This array is then sorted lexicographically, and the concatenation of the last character in each entry of the array is output.

\subsubsection{Dynamic Markov Coding}
Created by Cormack and Horspool \autocite{Cormack1987}. As with SR and PPM, this method of compression uses a predictor which assigns probabilities to potential next tokens, as well as arithmetic coding to assign shorter codes to more likely tokens. Unlike SR and PPM, this method compresses the input one bit at a time, rather than one byte at a time.

\subsubsection{Context Mixing}
Context mixing uses multiple predictors, each using a different context (or features of the text) and each producing its own probability distribution for the next symbol in its input, and combines these probability distributions through one of many possible averaging methods (e.g. linear, logistic) into a unified distribution which is often closer to the true distribution. This results in decoders which are less “surprised” by the next received symbol and which have accordingly assigned it a shorter code. To aid accuracy, CM is implemented in an “adaptive” way, meaning that the weights assigned to each sub-model’s predictions are re-evaluated based on how accurate they’ve been so far. \autocite{Mahoney2005}

\subsubsection{Long Short-Term Memory and Transformers}
LSTMs are a type of recurrent neural network which is fed its previous output as one of its inputs and which maintains an internal state indicating which of its previously seen inputs are relevant to future outputs. Using the attention mechanism, transformers can similarly select relevant parts of an input sequence as context for the part being encoded. Because of their ability to maintain context information, LSTMs and transformers are particularly suited for sequence prediction and so form the basis of another type of predictive coding.
\subsection{Known results}

In 2006, Hutter created a prize to “encourage development of intelligent compressors/programs as a path to AGI” \autocite{Hutter2006}. The challenge is to compress 1GB of text extracted from Wikipedia, chosen because “Wikipedia is an extensive snapshot of Human Knowledge. If you can compress the first 1GB of Wikipedia better than your predecessors, your (de)compressor likely has to be smart(er)” and explaining that “while intelligence is a slippery concept, file sizes are hard numbers.”

The prize is awarded based on improvements in data compression on a specific 1 GB text file, titled \texttt{enwik9}, which is extracted from the English Wikipedia.

The approaches taken by the record holders as well as the performance of their methods gives an indication of the current practical possibilities.

\todo[inline, color=yellow]{Include information about known records and their compression ratios}