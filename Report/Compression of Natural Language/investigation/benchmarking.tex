\section{Benchmarking compression algorithms}

For a rough bound on what's practically possible given commonly used tools, I start by analyzing the performance of various existing non-text-specific compression algorithms on the Project Gutenberg dataset. The result of this benchmarking is displayed in Table \ref{tab:compalg_benchmarks}.

\begin{table}[ht]
\centering
\begin{tabular}{ |p{3cm}||p{4cm}|p{3cm}|  }
 %\hline
 %\multicolumn{3}{|c|}{Compression algorithm benchmarks} \\
 \hline
 Algorithm & Size (bytes) & Compression ratio\\
 \hline
    LZMA & 20815738 & 3.745084\\
    bzip2 & 21055590 & 3.702422\\
    gzip & 28702450 & 2.716029\\
    zlib & 28840353 & 2.703042\\
    No compression & 77956688 & 1.000000\\
 \hline
\end{tabular}
\caption{Compression algorithm benchmarks
\label{tab:compalg_benchmarks}}
\end{table}

The best performing compression algorithm is the Lempel–Ziv–Markov chain algorithm (LZMA), with a compression ratio of 3.745. Given that the vast majority of characters in Gutenberg texts are represented in a single byte (except for unicode characters), this compression method indicates that, at most, each character in these texts contains on average 8/3.745 = 2.14 bits of information.

This figure is obtained empirically and establishes an upper bound on the amount of information per character in English text, but how do we establish the actual amount of information per character and thereby estimate the best compression ratio that can be hoped for? In other words, how do we calculate the entropy of English text?

%\subsection{Known results}

In 2006, Hutter created a prize to “encourage development of intelligent compressors/programs as a path to AGI” \autocite{Hutter2006}. The challenge is to compress 1GB of text extracted from Wikipedia, chosen because “Wikipedia is an extensive snapshot of Human Knowledge. If you can compress the first 1GB of Wikipedia better than your predecessors, your (de)compressor likely has to be smart(er)” and explaining that “while intelligence is a slippery concept, file sizes are hard numbers.”

The prize is awarded based on improvements in data compression on a specific 1 GB text file, titled \texttt{enwik9}, which is extracted from the English Wikipedia.

The approaches taken by the record holders as well as the performance of their methods gives an indication of the current practical possibilities.

\todo[inline, color=yellow]{Include information about known records and their compression ratios}
\todo[inline, color=yellow]{Potentially mention Hutter Prize compression benchmarks, see "known results" file}